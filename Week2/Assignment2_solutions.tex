\documentclass{article}

% Include necessary libraries
\usepackage{amsmath} % For mathematical symbols and equations
\usepackage{graphicx} % For including images
\usepackage{hyperref} % For hyperlinks
\usepackage{listings} % For including code snippets
\usepackage{amssymb}
\usepackage{qcircuit}
\usepackage{tikz}
% \usepackage{quantikz}
%for adjusting margins
\usepackage[margin=1in]{geometry}
\usepackage{mathtools}

\DeclarePairedDelimiter\bra{\langle}{\rvert}
\DeclarePairedDelimiter\ket{\lvert}{\rangle}
\DeclarePairedDelimiterX\braket[2]{\langle}{\rangle}{#1\,\delimsize\vert\,\mathopen{}#2}



% Set the title, author and date of the document
\title{Assignment 2 Solutions}
\author{Vishal Bysani}
\date{June 2024}

\begin{document}
\maketitle
\section{Quantum Gates}
\subsection{Question 1}
The Toffoli gate can be implemented using Hadamard, CNOT, $\pi$/8 and phase gates in  the following manner:
\[
\Qcircuit @C=1em @R=1em {
    & \qw  & \qw & \qw&\ctrl{2} & \qw & \qw & \qw &\ctrl{2} & \qw & \ctrl{1} & \qw & \ctrl{1} & \gate{T}\\
    & \qw  & \ctrl{1} & \qw & \qw &  \qw & \ctrl{1} & \qw & \qw&\gate{T^\dagger} & \targ & \gate{T^\dagger} & \targ & \gate{S}\\
    &  \gate{H} & \targ & \gate{T^\dagger} & \targ & \gate{T} & \targ & \gate{T^\dagger} & \targ & \gate{T} & \gate{H} & \qw & \qw\\
}
\]

\subsection{Question 2}
The Fredkin gate can be implemented using three Toffoli gates in the following manner:
\[
\Qcircuit @C=1em @R=1em {
    & \ctrl{1} & \ctrl{1} &  \ctrl{1} & \qw \\
    & \ctrl{1}    & \targ   & \ctrl{1} & \qw \\
    & \targ    & \ctrl{-1}    & \targ    & \qw
}
\]

\subsection{Question 3}
The Fredkin gate can be implemented using least number of Toffoli and $\pi/8$ gates in the following manner:
\[
\Qcircuit @C=1em @R=1em {
    &\qw&\ctrl{1}&\qw&\qw\\
    &\ctrl{1}&\targ{}&\ctrl{1}&\qw\\
    &\targ{}&\ctrl{-1}&\targ{}&\qw
    }
    \]

    If the first qubit is $0$, then the Toffoli just performs the identity, hence the CNOTs cancel leading to an overall
    identity. If the first qubit is $1$, then the Toffoli performs a CNOT on the last 2 qubits, which overall performs the
    SWAP operation.\\

Further, by taking $V = \frac{(1-i)(1+iX)}{2}$ which gives $V^2 = X$ , we can replace the Toffoli with:\\
\[
\Qcircuit @C=1em @R=1em {
    &\qw&\qw&\ctrl{1}&\qw&\ctrl{1}&\ctrl{2}&\qw&\qw\\
    &\ctrl{1}&\ctrl{1}&\targ{}&\ctrl{1}&\targ{}&\qw&\ctrl{1}&\qw\\
    &\targ&\gate{V}&\qw&\gate{V^\dagger}&\qw&\gate{V}&\targ&\qw
}\]
\end{document}